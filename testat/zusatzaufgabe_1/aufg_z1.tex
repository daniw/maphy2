% coding:utf-8

%testat_z1 Zusatzaufgabe zu Testat
%Copyright (C) 2013, Daniel Winz

%----------------------------------------

\section{Testat – Zusatzaufgabe 1: Das Längenproblem}
(3 Punkte, Abgabe 28.10.13)

Das Längenproblem limitierte lange Zeit die Genauigkeit der Navigation zu See. 
Im Jahre 1714 setzte das Englische Parlament ein Preisgeld von 20‘000 £ für 
eine Lösung des Problems  aus, welche besser als ½ Grad Abweichung erreichen 
könnte.

Eine Lösung des Längenproblems wurde schliesslich durch die sehr genauen 
mechanischen Uhren von John Harrison erbracht. Das in vierter Iteration 
verbesserte Modell (als H4 bezeichnet) erreichte auf der 81-tägigen Fahrt nach 
Jamaika und zurück nur eine Gangabweichung von 5 Sekunden, was der Genauigkeit 
von modernen Quarzuhren entspricht!

\subsection{Aufgaben:}
\begin{enumerate}
  \item Erläutern Sie kurz, wieso die Bestimmung der geographischen Länge, im 
        Gegensatz zur geographischen Breite, bis ins 18. Jahrhundert ein 
        grosses Problem darstellte.
  \item Stellen Sie weiter kurz dar, wie mit einer genauen Uhr das 
        Längenproblem gelöst werden kann und welche (relative und absolute) 
        Genauigkeit nötig wäre, um bei der Fahrt von London nach Jamaika in 
        40 ½ Tagen weniger als ½ Längengrad Abweichung zu erreichen.
  \item Nun wollen wir allgemein die Genauigkeit einer Uhr abschätzen. \\
        Hierzu betrachten wir eine Uhr als schwach gedämpfte Schwingung (z.B. 
        des Pendels oder der Unruh einer mechanischen Uhr), welche durch einen 
        geeigneten Mechanismus in Resonanz angeregt wird. Der Q-Faktor der 
        Schwingung bestimmt dabei bekanntlich die Breite der Resonanzkurve und 
        ist damit auch direkt ein Mass für die Genauigkeit der Uhr. Erläutern 
        Sie diesen Sachverhalt kurz. \\
        Hinweis: Was passiert bei schwacher Dämpfung Q >> 1 mit der Amplitude 
        der erzwungenen Schwingung, wenn die Anregungsfrequenz von der 
        Resonanzfrequenz des Oszillators abweicht?
  \item Wir betrachten nun eine mechanische Pendeluhr, welche eine 
        Schwingungsdauer von 1 Sekunde hat und deren Schwingungsamplitude (bei 
        Abschalten der Anregungen) innerhalb von ½ Stunde auf die Hälfte 
        abnimmt. Bestimmen Sie den Q-Faktor dieser Uhr und schätzen Sie die 
        absolute Genauigkeit über einen Tag ab. Könnten Sie mit einer Uhr 
        dieser Genauigkeit (müsste dann aber auch auf einem bewegten Schiff 
        funktionieren) die 20‘000 £ gewinnen?
\end{enumerate}

\subsubsection{Bedingungen:}
\begin{enumerate}
  \item Die maximal erreichbare Punktezahl (für alle Teilaufgaben zusammen) 
        ist 3 Punkte.
  \item Es wird eine schriftliche Ausführung (mit Skizzen und Berechnungen) im 
        Umfang von etwa einer DIN A4 Seite erwartet. Sie können auch einfach 
        eine handgeschriebene Seite scannen.
  \item Eine individuelle Lösung wird erwartet, d.h. im offensichtlichem Fall 
        einer Kopie werden keine Punkte erteilt (für keine der Varianten).
  \item Die Abgabe muss bis zum 28. Oktober 13:00 erfolgen.
  \item Die Abgabe erfolgt elektronisch unter folgender E-mail.
  \item Sie dürfen – es ist sogar erwünscht – mittels der Kommentare am Ende 
        der Blogseite über das Problem diskutieren, um z.B. Tipps und 
        Hilfestellungen austauschen. Beachten Sie aber den Punkt 3.
\end{enumerate}