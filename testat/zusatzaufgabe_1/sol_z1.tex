% coding:utf-8

%testat_z1 Zusatzaufgabe zu Testat
%Copyright (C) 2013, Daniel Winz

%----------------------------------------

\subsection{Lösung}
Die Bestimmung der geografischen Breite kann über die Lage der Sterne bestimmt 
werden. Da sich die Erde dreht bewegen sich die Sterne scheinbar in Richtung 
der geografischen Länge. Daher kann die geografische Länge mit diesem Verfahren 
nur bestimmt werden, wenn die momentane Lage der Erde bekannt ist. Dafür ist es 
notwendig, die Uhrzeit an einem fixen Ort auf der Erde zu kennen. Bis ins 18. 
Jahrhundert waren Uhren mit der notwendigen Genauigkeit nicht verfügbar. 

Wird nun eine genaue Uhr mitgeführt, kann die Position der Sterne gemessen und 
mit der berechneten Position für den Startort verglichen werden. Die Differenz 
entspricht dabei der Differenz der geografischen Länge zum Startpunkt. 

Soll bei einer Fahrt von London nach Jamaika, welche $40\frac{1}{2}$ Tage 
dauert, die Abweichung weniger als $\frac{1}{2}$ Längengrad betragen, so muss 
die Uhr folgende Genauigkeit aufweisen: 

Die Erde dreht sich in 24 Stunden einmal um die eigene Achse. Dabei 
überstreicht sie $360^\circ$. Ein Fehler von $\frac{1}{2}^\circ$ entspricht daher 
einer Zeitdifferenz von 120 Sekunden. 
\[ \frac{24h \cdot 3600\frac{s}{h}}{720} = 120 s \]
Bei einer Reisezeit von $40\frac{1}{2}$ Tagen entspricht das einer relative 
Genauigkeit von: 
\[ \frac{120 s}{40.5 d \cdot 86400 \frac{s}{d}} = 34.29 \cdot 10^{-6} 
= 0.0034\% \]